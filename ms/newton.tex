\documentclass[12pt]{article}
\usepackage{verbatim, amsmath,amssymb,amsthm,graphicx}
\usepackage[margin=1in,nohead]{geometry}
\usepackage{sectsty}
\usepackage{float}
\usepackage{natbib}
\usepackage[usenames]{color}
\usepackage{xspace}
\usepackage{subfig}
\usepackage{bbm}
\usepackage{bm}

\sectionfont{\normalsize}
\subsectionfont{\small}

\title{}
\date{}
\author{}
\newtheorem{theorem}{Theorem}[section]
\newtheorem{definition}{Definition}[section]
\newtheorem{example}{Example}[section]

\newcommand{\Remark}[2]{{\color{red}Remark from #1: #2}\xspace}
\newcommand{\argmin}[1]{\underset{#1}{\operatorname{argmin}}\text{ }}
\newcommand{\argmax}[1]{\underset{#1}{\operatorname{argmax}}\text{ }}
\newcommand{\minimax}[2]{\argmin{#1}\underset{#2}{\operatorname{max}}}
\newcommand{\bb}{\textbf{b}}

\newcommand{\Var}{\text{Var }}
\newcommand{\Cov}{\text{Cov }}
\newcommand{\E}{\mathbb{E}}
\newcommand{\ind}{\mathbbm{1}}

\newcommand{\V}[1]{{\bm{\mathbf{\MakeLowercase{#1}}}}} % vector
\newcommand{\M}[1]{{\bm{\mathbf{\MakeUppercase{#1}}}}} % matrix

\newcommand{\todo}[1]{{\color{red}TODO: #1}}

\newenvironment{my_enumerate}{
  \begin{enumerate}
    \setlength{\itemsep}{1pt}
    \setlength{\parskip}{0pt}
    \setlength{\parsep}{0pt}}{\end{enumerate}
}



% Alter some LaTeX defaults for better treatment of figures:
% See p.105 of [yas] elisp error!TeX Unbound'' for suggested values.
% See pp. 199-200 of Lamport's [yas] elisp error!LaTeX'' book for details.
%   General parameters, for ALL pages:
\renewcommand{\topfraction}{0.9}% max fraction of floats at top
\renewcommand{\bottomfraction}{0.8}% max fraction of floats at bottom
%   Parameters for TEXT pages (not float pages):
\setcounter{topnumber}{2}
\setcounter{bottomnumber}{2}
\setcounter{totalnumber}{4}     % 2 may work better
\setcounter{dbltopnumber}{2}    % for 2-column pages
\renewcommand{\dbltopfraction}{0.9}% fit big float above 2-col. text
\renewcommand{\textfraction}{0.07}% allow minimal text w. figs
%   Parameters for FLOAT pages (not text pages):
\renewcommand{\floatpagefraction}{0.7}% require fuller float pages
% N.B.: floatpagefraction MUST be less than topfraction !!
\renewcommand{\dblfloatpagefraction}{0.7}% require fuller float pages

% remember to use [htp] or [htpb] for placement





\begin{document}
\noindent
\textbf{James Long}\\
\textbf{\today}\\
\textbf{Description of Model and Newton Optimization Algorithm}

The data is $(m_i,t_i)$ for $i=1,\ldots,n$ and the parameters are $(\omega,a,\phi,\beta,\sigma^2)$. The model is
\begin{equation*}
m_i = a\gamma(\omega t_i + \phi) + \beta + \epsilon_i
\end{equation*}
where $\epsilon_i \sim N(0,\sigma^2)$ independent. The function $\gamma$ is 

\begin{equation*}
  \gamma(t) =    \left\{
  \begin{array}{lr}
    1 - \frac{2}{c}t &:  t \text{ mod } 1 > c\\ 
    \frac{2}{1-c}t - \frac{1+c}{1-c}  &: o.w.
  \end{array} \right.
\end{equation*}

We seek the ML estimator
\begin{equation*}
\widehat{\omega},\widehat{\beta},\widehat{a},\widehat{\phi} = \argmin{\omega,\beta,a,\phi} \sum_{i=1}^n (m_i - a\gamma(\omega t_i + \phi) - \beta_0)^2 \equiv \argmin{\omega,\beta,a,\phi} g_\omega(\beta,a,\phi)
\end{equation*}
We propose a grid search across $\omega$ with Newton updates for $\theta = (\beta,a)$. We derive the Hessian and the gradient for the Newton updates. 

The gradient is
\begin{equation*}
\nabla = \begin{pmatrix}
\frac{\partial g}{\partial \theta} \\
\frac{\partial g}{\partial \phi}
\end{pmatrix} \in \mathbb{R}^{3 \times 1}
\end{equation*}
The Hessian is
\begin{equation*}
H = \begin{pmatrix}
\frac{\partial g^2}{\partial \theta^2} & \frac{\partial g^2}{\partial \theta \partial \phi} \\
\frac{\partial g^2}{\partial \theta \partial \phi}^T  & \frac{\partial g^2}{\partial \phi^2} 
\end{pmatrix} \in \mathbb{R}^{3 \times 3}
\end{equation*}



Now define
\begin{equation*}
X = \begin{pmatrix}
1 & \gamma(\omega t_1 + \phi) \\
\vdots & \vdots\\
1 & \gamma(\omega t_n + \phi)
\end{pmatrix} \in \mathbb{R}^{n \times 2}
\end{equation*}

So
\begin{equation*}
g_\omega(\beta,a,\phi) = g_\omega(\theta,\phi) = (m - X\theta)^T(m - X\theta)
\end{equation*}
Therefore
\begin{equation*}
\frac{\partial g}{\partial \theta} = -2X^Tm + 2X^TX\theta
\end{equation*}
and
\begin{equation*}
\frac{\partial g^2}{\partial \theta^2} = 2X^TX.
\end{equation*}
For $\phi$ we have
\begin{equation*}
\frac{\partial g}{\partial \phi} = -2a(m - 1\beta)^T\gamma' + 2a^2\gamma^T\gamma'
\end{equation*}
and
\begin{equation*}
\frac{\partial g^2}{\partial \phi^2} = 2a^2\gamma^{'T}\gamma'
\end{equation*}
For the two mixed derivates we get
\begin{equation*}
\frac{\partial g^2}{\partial a \partial \phi} = -2(m-1\beta)^T\gamma' + 4a\gamma^{'T}\gamma
\end{equation*}
and
\begin{equation*}
\frac{\partial g^2}{\partial \beta \partial \phi} = 2a\gamma^T1
\end{equation*}
Finally note that
\begin{equation*}
  \gamma'(t\omega + \phi) =    \left\{
  \begin{array}{lr}
     - \frac{2}{c} &:  t\omega + \phi \text{ mod } 1 > c\\ 
    \frac{2}{1-c}  &: o.w.
  \end{array} \right.
\end{equation*}
and
\begin{equation*}
  \gamma''(t\omega + \phi) =    0
\end{equation*}





%\bibliographystyle{plainnat}
%\bibliography{refs}


\end{document}

