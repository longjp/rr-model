\documentclass[12pt]{article}
\usepackage{verbatim, amsmath,amssymb,amsthm,graphicx}
\usepackage[margin=1in,nohead]{geometry}
\usepackage{sectsty}
\usepackage{float}
\usepackage{natbib}
\usepackage[usenames]{color}
\usepackage{xspace}
\usepackage{subfig}
\usepackage{bbm}
\usepackage{bm}

\sectionfont{\normalsize}
\subsectionfont{\small}

\title{}
\date{}
\author{}
\newtheorem{theorem}{Theorem}[section]
\newtheorem{definition}{Definition}[section]
\newtheorem{example}{Example}[section]

\newcommand{\Remark}[2]{{\color{red}Remark from #1: #2}\xspace}
\newcommand{\argmin}[1]{\underset{#1}{\operatorname{argmin}}\text{ }}
\newcommand{\argmax}[1]{\underset{#1}{\operatorname{argmax}}\text{ }}
\newcommand{\minimax}[2]{\argmin{#1}\underset{#2}{\operatorname{max}}}
\newcommand{\bb}{\textbf{b}}

\newcommand{\Var}{\text{Var }}
\newcommand{\Cov}{\text{Cov }}
\newcommand{\E}{\mathbb{E}}
\newcommand{\ind}{\mathbbm{1}}

\newcommand{\V}[1]{{\bm{\mathbf{\MakeLowercase{#1}}}}} % vector
\newcommand{\M}[1]{{\bm{\mathbf{\MakeUppercase{#1}}}}} % matrix

\newcommand{\todo}[1]{{\color{red}TODO: #1}}

\newenvironment{my_enumerate}{
  \begin{enumerate}
    \setlength{\itemsep}{1pt}
    \setlength{\parskip}{0pt}
    \setlength{\parsep}{0pt}}{\end{enumerate}
}



% Alter some LaTeX defaults for better treatment of figures:
% See p.105 of [yas] elisp error!TeX Unbound'' for suggested values.
% See pp. 199-200 of Lamport's [yas] elisp error!LaTeX'' book for details.
%   General parameters, for ALL pages:
\renewcommand{\topfraction}{0.9}% max fraction of floats at top
\renewcommand{\bottomfraction}{0.8}% max fraction of floats at bottom
%   Parameters for TEXT pages (not float pages):
\setcounter{topnumber}{2}
\setcounter{bottomnumber}{2}
\setcounter{totalnumber}{4}     % 2 may work better
\setcounter{dbltopnumber}{2}    % for 2-column pages
\renewcommand{\dbltopfraction}{0.9}% fit big float above 2-col. text
\renewcommand{\textfraction}{0.07}% allow minimal text w. figs
%   Parameters for FLOAT pages (not text pages):
\renewcommand{\floatpagefraction}{0.7}% require fuller float pages
% N.B.: floatpagefraction MUST be less than topfraction !!
\renewcommand{\dblfloatpagefraction}{0.7}% require fuller float pages

% remember to use [htp] or [htpb] for placement





\begin{document}
\noindent
\textbf{James Long}\\
\textbf{\today}\\
\textbf{Summary of Model and Absolute Magnitude--Period Dependence}


\section{Model}

Let $m_{ib}(t)$ be the magnitude of light curve $i$ in band $b$ at time $t$. Our model is:
\begin{equation*}
m_{ib}(t) = \mu_i + M_{b}(p_i) + \text{E[B-V]}_iR_b + a_i\gamma_b(t/p_i + \phi_i)
\end{equation*}
where the \underline{lightcurve specific} terms (indexed by $i$) are 
\begin{align*}
  \mu_i &= \text{ distance modulus }\\
  \text{E[B-V]}_i &= \text{ extinction due to dust }\\
  a_i &= \text{amplitude, shared across all bands}\\
  p_i &= \text{period}\\
  \phi_i &= \text{phase}
\end{align*}
and the \underline{global parameters} common to all RRL are
\begin{align*}
  M_{b}(p) &= \text{ absolute magnitude in band $b$, dependent on period } p\\
  R_b &= \text{ extinction law parameter in band $b$}\\
  \gamma_b &= \text{shape of lightcurve in band $b$}
\end{align*}
$R_b$ is considered known. $\gamma_b$ is a mean $0$, period $1$ function, i.e.
\begin{align*}
  &\int_0^1 \gamma_b(t) dt = 0 \text{ for all } b\\
  &\gamma_b(t) = \gamma_b(t + 1) \text{ for all } t \text{ and } b.
\end{align*}
This note concerns how we model the absolute magnitude terms $M_b(p)$. Let $\bar{m}_{ib}$ be the mean absolute magnitude for RRL $i$ in band $b$, specifically
\begin{equation*}
  \bar{m}_{ib} = \int_0^p m_{ib}(t) dt.
\end{equation*}
Since the $\gamma_b$ terms are mean $0$ we have
\begin{equation*}
\bar{m}_{ib} = \mu_i + M_{b}(p) + \text{E[B-V]}_iR_b
\end{equation*}
$M_{b}(p)$ is assumed to be a quadratic function of log period in each band. Let $x_i = \log_{10}(p_i + 0.2)$. Specifically
\begin{equation*}
  M_{b}(p) = \beta_{0b} + \beta_{1b}x_i  + \beta_{2b}x_i^2
\end{equation*}

In Section \ref{sec:nonident} we discuss some degeneracies among the model parameters which prevent a straigtforward estimation of all terms. In Section \ref{sec:compare} we compare performance (period, distance, dust estimation accuracy) of two sets of $\beta$ parameters. In Section \ref{sec:questions} we address some questions for going forward.

\section{Degeneracies}
\label{sec:nonident}

There are degeneries in the model in the sense that different sets of parameters will produce the same RRL light curves. Thus using only RRL light curve data, one cannot uniquely determine the parameters. This is called non--identifiability in statistics. The following are a few degeneracies (possibly more not specified):
\begin{enumerate}
\item $\mu_i$ and $\beta_{0b}$: We can increase the distance modulli by any fixed amount $c$ by decreasing all the $\beta_{0b}$ by $c$. This will produce the same light curves because:
  \begin{equation*}
    \mu_i + \beta_{0b} = (\mu_i + c) + (\beta_{0b}-c)
  \end{equation*}
  Since the distance to $RRL$ $i$ (in parsecs) is
  \begin{equation*}
    d_i = 10^{\mu_i/5 + 1},
  \end{equation*}
  increasing $\mu_i$ by $c$ will result in a distance that is off by $10^{c/5}$ because
  \begin{equation*}
   10^{(\mu_i + c)/5 + 1} = (10^{\mu_i/5 + 1})10^{c/5} = d_i10^{c/5}.
  \end{equation*}
  Thus systematic biases in $d_i$ may be due to incorrect $\beta_{0b}$ values.
\item $E[B-V]_i$ and $\beta_{0}$: Keeping all other terms constant and letting $c$ be any real number:
  \begin{equation*}
    \beta_{0b} + E[B-V]_iR_b = (\beta_{0b} + cR_b) + (E[B-V]_i-c)R_b
  \end{equation*}
  Thus systematic biases in $E[B-V]_i$ may be due to incorrect $\beta_{0b}$ values.
\item $\mu$ and $\beta_{1b}$,$\beta_{2b}$: One can alter all of the $\mu_i$ and all of the $\beta_{1b}$ or $\beta_{2b}$. Specifically for any $c$:
  \begin{equation*}
    \mu_i + \beta_{1b}x_i = (\mu_i + cx_i) + (\beta_{1b}-c)x_i
  \end{equation*}
  This would produce distance modulli which are biased in the ``direction''
  \begin{equation*}
    x_i=\log_{10}(p_i+0.2).
  \end{equation*}
  The same degeneracy holds for $\beta_{02}$ and $x_i^2$.
\end{enumerate}

\section{Comparison}
\label{sec:compare}

For now, we put aside the degeneracy issues of Section \ref{sec:nonident} and fix the values of $\beta_{0b}$, $\beta_{1b}$, and $\beta_{2b}$. Once these values have been fixed, all degeneracies are gone and we can estimate the $\gamma_b$ values using well--sampled RRL data. Then for any given light curve we can estimate the vector of parameters $(\mu_i,\text{E[B-V]}_i,a_i,p_i,\phi_i)$. We obtain RRL parameter fits using two sets of $\beta$ parameters:
\begin{itemize}
\item \underline{Period Dependent Abs Mag:} The file \texttt{rrmag.dat} on the project website contains a set of $\beta$ values. When fitting for the $5$ RRL specific parameter on a grid of periods, the absolute magnitude is computed for each period in the grid and subtracted off the observed magnitudes.
\item \underline{Fixed Absolute Mag (Old Templates):} This is the method Katelyn is currently using. The median period for a set a few hundred well--sampled RRL is $\approx 0.59$. Using the \texttt{rrmag.dat} $\beta$ values we compute
  \begin{equation*}
    M_b(0.59) = (u=1.73,g=0.64,r=0.49,i=0.48,z=0.51)
  \end{equation*}
  This is the absolute magnitude used for fitting at \underline{all} periods in the grid.
\end{itemize}
The fixed absolute magnitude method was used because the original fitting code could only use fixed absolute magnitudes. The choice of $M_b(0.59)$ was used because this was the absolute magnitude for a typical RRL. The updated code can use any period, absolute magnitude relationship (fixed or not).

We fit these two templates on 1) well--sampled SDSS Stripe 82 RRL and 2) downsampled SDSS Stripe 82 RRL. We compare the fits based on three metrics:
\begin{enumerate}
\item \underline{Period estimation:} Model estimated periods are compared to Sesar 2010 reported periods. Sesar 2010 periods are all very close to correct.
\item \underline{Distance:} Model estimated distances are compared to the Sesar 2010 distance estimates. Sesar discusses distance determination and uncertainty in Section 4.1. ``The estimated fractional error in the heliocentric distance is 0.06.''
\item \underline{Dust:} Model estimated extinction in r--band ($\text{E[B-V]}_iR_r$) is compared to values reported by Sesar 2010 (Table 3) which are taken from Schlegel et. al. 1998.
\end{enumerate}

\section{Questions}
\label{sec:questions}

\begin{enumerate}
\item circularity of argument in terms of estimating distance parameters
\item issue with estimating distance from mean mag or after fitting model
\end{enumerate}



%\bibliographystyle{plainnat}
%\bibliography{refs}


\end{document}

