%%%%%%%%%%%%%%%%%%%%%%%%%%%%%%%%%%%%%%%%%
% a0poster Portrait Poster
% LaTeX Template
% Version 1.0 (22/06/13)
%
% The a0poster class was created by:
% Gerlinde Kettl and Matthias Weiser (tex@kettl.de)
% 
% This template has been downloaded from:
% http://www.LaTeXTemplates.com
%
% License:
% CC BY-NC-SA 3.0 (http://creativecommons.org/licenses/by-nc-sa/3.0/)
%
%%%%%%%%%%%%%%%%%%%%%%%%%%%%%%%%%%%%%%%%%

%----------------------------------------------------------------------------------------
%	PACKAGES AND OTHER DOCUMENT CONFIGURATIONS
%----------------------------------------------------------------------------------------

\documentclass[a0,portrait]{a0poster}

\usepackage{multicol} % This is so we can have multiple columns of text side-by-side
\columnsep=100pt % This is the amount of white space between the columns in the poster
\columnseprule=3pt % This is the thickness of the black line between the columns in the poster

\usepackage{algorithm,algpseudocode,ifpdf}


\usepackage[svgnames]{xcolor} % Specify colors by their 'svgnames', for a full list of all colors available see here: http://www.latextemplates.com/svgnames-colors

\usepackage{times} % Use the times font
%\usepackage{palatino} % Uncomment to use the Palatino font

\usepackage{graphicx} % Required for including images
\graphicspath{{figures/}} % Location of the graphics files
\usepackage{booktabs} % Top and bottom rules for table
\usepackage[font=small,labelfont=bf]{caption} % Required for specifying captions to tables and figures
\usepackage{amsfonts, amsmath, amsthm, amssymb} % For math fonts, symbols and environments
\usepackage{wrapfig} % Allows wrapping text around tables and figures

\newcommand{\argmin}[1]{\underset{#1}{\operatorname{argmin}}\text{ }}
\newcommand{\argmax}[1]{\underset{#1}{\operatorname{argmax}}\text{ }}


\begin{document}

%----------------------------------------------------------------------------------------
%	POSTER HEADER 
%----------------------------------------------------------------------------------------

% The header is divided into two boxes:
% The first is 75% wide and houses the title, subtitle, names, university/organization and contact information
% The second is 25% wide and houses a logo for your university/organization or a photo of you
% The widths of these boxes can be easily edited to accommodate your content as you see fit

\begin{minipage}[b]{0.75\linewidth}
\veryHuge \color{NavyBlue} \textbf{Near Field Cosmology with RR Lyrae Variables} \color{Black}\\ % Title
\Huge\textit{Building and Testing Parsimoneous Models}\\[2cm] % Subtitle
\huge \textbf{James Long \& Jennifer Marshall \& Lucas Macri}\\[0.5cm] % Author(s)
\huge Texas A\&M University, Department of Statistics and Physics \& Astronomy\\[0.4cm] % University/organization
\Large \texttt{jlong@stat.tamu.edu}\\% --- 1 (000) 111 1111\\
\end{minipage}
%
\begin{minipage}[b]{0.25\linewidth}
\includegraphics[width=20cm]{tamu.png}\\
\end{minipage}

\vspace{1cm} % A bit of extra whitespace between the header and poster content

%----------------------------------------------------------------------------------------

\begin{multicols}{2} % This is how many columns your poster will be broken into, a portrait poster is generally split into 2 columns

%% %----------------------------------------------------------------------------------------
%% %	ABSTRACT
%% %----------------------------------------------------------------------------------------

%% \color{Navy} % Navy color for the abstract

%% \begin{abstract}

%% Sed fringilla tempus hendrerit. Vestibulum ante ipsum primis in faucibus orci luctus et ultrices posuere cubilia Curae; Etiam ut elit sit amet metus lobortis consequat sit amet in libero. Lorem ipsum dolor sit amet, consectetur adipiscing elit. Phasellus vel sem magna. Nunc at convallis urna. isus ante. Pellentesque condimentum dui. Etiam sagittis purus non tellus tempor volutpat. Donec et dui non massa tristique adipiscing. Quisque vestibulum eros eu. Phasellus imperdiet, tortor vitae congue bibendum, felis enim sagittis lorem, et volutpat ante orci sagittis mi. Morbi rutrum laoreet semper. Morbi accumsan enim nec tortor consectetur non commodo nisi sollicitudin. Proin sollicitudin. Pellentesque eget orci eros. Fusce ultricies, tellus et pellentesque fringilla, ante massa luctus libero, quis tristique purus urna nec nibh.

%% \end{abstract}

%----------------------------------------------------------------------------------------
%	INTRODUCTION
%----------------------------------------------------------------------------------------

\section*{Motivation}

\begin{itemize}
\item Cosmological simulations predict the existence of many dwarf satellite galaxies.
\item Few Milky Way satellite galaxies have been observed: \textbf{Missing satellites problem} \cite{kauffmann1993formation,klypin1999missing,moore1999dark}
\item Finding satellite galaxies is difficult due to low surface brightness and diffuse nature.
\item Distance determination to structures in the Milky Way halo is difficult.
\item Finding RR Lyrae in halo enables distance determination, detection of structure. \cite{baker2015charting}
\end{itemize}



\section*{RR Lyrae in the Dark Energy Survey}

\begin{itemize}
\item Easy to estimate parameters for well observed RR Lyrae. For example,

\begin{center}\vspace{1cm}
\includegraphics[width=0.49\linewidth]{unfolded.pdf}
\includegraphics[width=0.49\linewidth]{folded.pdf}
\captionof{figure}{An SDSS Stript 82 RR Lyrae unfolded (left) and folded (right) observed in bands $g,i,r,u,z$.}
\end{center}\vspace{1cm}


\item The Dark Energy Survey will collect sparsely sampled RR Lyrae.

\begin{center}\vspace{1cm}
\includegraphics[width=0.45\linewidth]{unfolded_panstarrs.pdf}
\captionof{figure}{A PanSTARRS RR Lyrae. The Dark Energy Survey (DES) is collecting similar quality light curves. \cite{hernitschek2016finding} \label{fig:panstarrs}}
\end{center}\vspace{1cm}


\item Distance determination using RR Lyrae requires identifying light curves as RR Lyrae and estimating parameters such as period. This is difficult with light curves sampled at the quality of Figure \ref{fig:panstarrs}.


\end{itemize}


\section*{RR Lyrae Model Constructed with SDSS Stripe 82 Data}

\begin{itemize}
\item Proposed model: The brightness in band $b$ at time $t$ is

{\Large
\begin{equation*}
m_b(t) = \alpha + \beta_b + E[B-V]\delta_b + a\gamma_b(\omega t + \phi)
\end{equation*}
}

where the \underline{global parameters} common for all RR Lyrae are
\begin{align*}
\beta_b &=  \text{ magnitude off--set in band $b$ }\\
\delta_b &= \text{ attenuation caused by dust in band $b$ }\\
\gamma_b &= \text{ normalized shape of light curve in band $b$ }\\
\end{align*}
and the \underline{individual parameters} specific to each RR Lyrae are
\begin{align*}
\alpha &= \text{ mean brighness across all bands }\\
E[B-V] &= \text{ amount of dust }\\
a &= \text{ amplitude of variation }\\
\omega &= \text{ frequency of variation }\\
\phi &= \text{ phase of variation }
\end{align*}

This is a \underline{parsimoneous} model because it only has five free parameters to fit to each star.

\item For a specific RR Lyrae the data is $\{\{(m_{jb},t_{jb},\sigma_{jb})\}_{j=1}^{n_b}\}_{b=1}^B$ where $B$ is the number of filters and $m_{jb}$ is the star brightness measured at time $t_{jb}$ with uncertainty $\sigma_{jb}$ in band $b$. The data is connected to the model through
\begin{equation*}
m_{jb} = m_b(t_{jb}) + \epsilon_{jb}
\end{equation*}
where $\epsilon_{jb} \sim N(0,\sigma_{jb}^2)$. 

\item The \underline{global parameters} are estimated from a set of 309 well--observed SDSS Stripe 82 RR Lyrae AB \cite{sesar2010light}. For example, the estimated $\gamma_b$ functions are shown in Figure \ref{fig:gamma}.

\begin{center}\vspace{1cm}
\includegraphics[width=0.7\linewidth]{templates.pdf}
\captionof{figure}{$\gamma_b$ functions for bands $b=g,i,r,u,z$. \label{fig:gamma}}
\end{center}\vspace{1cm}


\end{itemize}

%%% results table

\section*{Fitting Model to Light Curve}

\begin{itemize}

\item Define

\begin{equation*}
g(\omega,\alpha,E[B-V],a,\phi) = \sum_{j=1}^{n_b}\left(\frac{m_{jb} - \alpha - E[B-V]d_b - a\gamma_b(\omega t_{jb} + \phi)}{\sigma_{bi}}\right)^2
\end{equation*}

\item Estimate parameters through maximum likelihood:
\begin{itemize}
\item Likelihood is highly multimodal in $\omega$, so perform grid search on frequency.
\item Model is linear in $\alpha$,$E[B-V]$, and $a$, so closed form updates.
\item Newton--Raphson updates for $\phi$ parameter.
\end{itemize}

\item Pseduocode

\begin{algorithmic}[1]
  \State $\Omega \gets (\omega_1,\ldots,\omega_N)$ \Comment{Grid for Frequency}
  \State $\phi \gets Unif[0,1]$ \Comment{Random Initial Phase}
  \For{$k = 1,\ldots,N$}
  \For{$l = 1,\ldots,10$}
  \State $\alpha,E[B-V],d \gets \argmin{\alpha,E[B-V],a} g(\omega,\alpha,E[B-V],a,\phi)$  \Comment{Closed form solutions}
  \State $\phi = \phi - \left[\frac{d^2}{d\phi^2}g(\omega_k,\alpha,E[B-V],a,\phi)\right]^{-1}\left[\frac{d}{d\phi}g(\omega_k,\alpha,E[B-V],a,\phi)\right] $ \Comment{Newton update for $\phi$}
  %% \State $p \gets p + 1$
  %% %% \Until{$||\theta_{l,p} - \theta_{l,p-1}|| < \epsilon$}
  %% \State $\theta_{l} \gets \theta_{l,p}$
  \EndFor
  \State $\text{RSS}(\omega_k) = g(\omega_k,\alpha,E[B-V],a,\phi)$
  \EndFor
  \State $\widehat{\omega} = \min \text{RSS}(\omega)$
\end{algorithmic}


%% \begin{equation*}
%% \widehat{\omega} = \argmin{\omega} \min_{\phi} \min_{\alpha,E[B-V],a} \sum_{b=1}^B \sum_{j=1}^{n_b}\left(\frac{m_{jb} - \alpha - E[B-V]d_b - a\gamma_b(\omega t_{jb} + \phi)}{\sigma_{bi}}\right)^2
%% \end{equation*}

%% \item For fixed $\omega$ and $\phi$, model is linear, so closed for solutions for parameters $\alpha,E[B-V],a$.

%% \item Likelihood is highly multimodel in $\omega$ parameter, so perform grid search on $\omega$.

%% \item At fixed $\omega,\alpha,E[B-V],a$, update $\phi$ using Newton Raphson: warm start

%% \begin{align*}
%% \phi^{(m)} &= \\
%% H(g) &=
%% \end{align*}

\end{itemize}


\section*{Results}

\begin{itemize}
\item Example of Model Fit to PanSTARRS light curve.

\begin{center}\vspace{1cm}
\includegraphics[width=0.6\linewidth]{unfolded_panstarrs.pdf}
\captionof{figure}{PanSTARRS RR Lyrae with Model Fit\cite{hernitschek2016finding} \label{fig:panstarrs_fit}}
\end{center}\vspace{1cm}

\item We conduct a simulation study to compare model performance to the Generalized Lomb Scargle (GLS) for estimating periods. We downsampled a set of 307 SDSS-III Stripe 82 RR Lyrae to 10 observations per band and then computed the fraction of time each method estimated the period correctly (within $0.2\%$) or in the top five period estimates.

\begin{center}
\begin{wraptable}{c}{12cm} % Left or right alignment is specified in the first bracket, the width of the table is in the second
\begin{tabular}{l l l}
\toprule
\textbf{Method} & \textbf{Fraction Top Period} & \textbf{Fraction Top 5}\\
\midrule
Proposed Model  & 0.0003262 & 0.562 \\
Generalized Lomb Scargle & 0.0015681 & 0.910 \\
\bottomrule
\end{tabular}
\captionof{table}{\color{Green} Table caption}
\end{wraptable}
\end{center}

%% \begin{wraptable}{l}{12cm} % Left or right alignment is specified in the first bracket, the width of the table is in the second
%% \begin{tabular}{l l l}
%% \toprule
%% \textbf{Method} & \textbf{Fraction Top Period} & \textbf{Fraction Top 5}\\
%% \midrule
%% Proposed Model  & 0.0003262 & 0.562 \\
%% Generalized Lomb Scargle & 0.0015681 & 0.910 \\
%% \bottomrule
%% \end{tabular}
%% \captionof{table}{\color{Green} Table caption}
%% \end{wraptable}



\end{itemize}

\section*{Ongoing Work}

\begin{itemize}
\item Quantify uncertainty in parameter estimates.
\item Classification of RR Lyrae.
\item Mapping of Halo Structures, detection of satellites.
\end{itemize}




%----------------------------------------------------------------------------------------
%	INTRODUCTION
%----------------------------------------------------------------------------------------

\color{SaddleBrown} % SaddleBrown color for the introduction

\section*{Introduction}

Aliquam non lacus dolor, \textit{a aliquam quam} \cite{Smith:2012qr}. Cum sociis natoque penatibus et magnis dis parturient montes, nascetur ridiculus mus. Nulla in nibh mauris. Donec vel ligula nisi, a lacinia arcu. Sed mi dui, malesuada vel consectetur et, egestas porta nisi. Sed eleifend pharetra dolor, et dapibus est vulputate eu. \textbf{Integer faucibus elementum felis vitae fringilla.} In hac habitasse platea dictumst. Duis tristique rutrum nisl, nec vulputate elit porta ut. Donec sodales sollicitudin turpis sed convallis. Etiam mauris ligula, blandit adipiscing condimentum eu, dapibus pellentesque risus.

\textit{Aliquam auctor}, metus id ultrices porta, risus enim cursus sapien, quis iaculis sapien tortor sed odio. Mauris ante orci, euismod vitae tincidunt eu, porta ut neque. Aenean sapien est, viverra vel lacinia nec, venenatis eu nulla. Maecenas ut nunc nibh, et tempus libero. Aenean vitae risus ante. Pellentesque condimentum dui. Etiam sagittis purus non tellus tempor volutpat. Donec et dui non massa tristique adipiscing.

%----------------------------------------------------------------------------------------
%	OBJECTIVES
%----------------------------------------------------------------------------------------

\color{DarkSlateGray} % DarkSlateGray color for the rest of the content

\section*{Main Objectives}

\begin{enumerate}
\item Lorem ipsum dolor sit amet, consectetur.
\item Nullam at mi nisl. Vestibulum est purus, ultricies cursus volutpat sit amet, vestibulum eu.
\item Praesent tortor libero, vulputate quis elementum a, iaculis.
\item Phasellus a quam mauris, non varius mauris. Fusce tristique, enim tempor varius porta, elit purus commodo velit, pretium mattis ligula nisl nec ante.
\item Ut adipiscing accumsan sapien, sit amet pretium.
\item Estibulum est purus, ultricies cursus volutpat
\item Nullam at mi nisl. Vestibulum est purus, ultricies cursus volutpat sit amet, vestibulum eu.
\item Praesent tortor libero, vulputate quis elementum a, iaculis.
\end{enumerate}

%----------------------------------------------------------------------------------------
%	MATERIALS AND METHODS
%----------------------------------------------------------------------------------------

\section*{Materials and Methods}

Fusce magna risus, molestie ut porttitor in, consectetur sed mi. Vestibulum ante ipsum primis in faucibus orci luctus et ultrices posuere cubilia Curae; Pellentesque consectetur blandit pellentesque. Sed odio justo, viverra nec porttitor vel, lacinia a nunc. Suspendisse pulvinar euismod arcu, sit amet accumsan enim fermentum quis. In id mauris ut dui feugiat egestas. Vestibulum ac turpis lacinia nisl commodo sagittis eget sit amet sapien.

%------------------------------------------------

\subsection*{Mathematical Section}

Nulla vel nisl sed mauris auctor mollis non sed. 

\begin{equation}
E = mc^{2}
\label{eqn:Einstein}
\end{equation}

Curabitur mi sem, pulvinar quis aliquam rutrum. (1) edf (2)
, $\Omega=[-1,1]^3$, maecenas leo est, ornare at. $z=-1$ edf $z=1$ sed interdum felis dapibus sem. $x$ set $y$ ytruem. 
Turpis $j$ amet accumsan enim $y$-lacina; 
ref $k$-viverra nec porttitor $x$-lacina. 

Vestibulum ac diam a odio tempus congue. Vivamus id enim nisi:

\begin{eqnarray}
\cos\bar{\phi}_k Q_{j,k+1,t} + Q_{j,k+1,x}+\frac{\sin^2\bar{\phi}_k}{T\cos\bar{\phi}_k} Q_{j,k+1} &=&\nonumber\\ 
-\cos\phi_k Q_{j,k,t} + Q_{j,k,x}-\frac{\sin^2\phi_k}{T\cos\phi_k} Q_{j,k}\label{edgek}
\end{eqnarray}
and
\begin{eqnarray}
\cos\bar{\phi}_j Q_{j+1,k,t} + Q_{j+1,k,y}+\frac{\sin^2\bar{\phi}_j}{T\cos\bar{\phi}_j} Q_{j+1,k}&=&\nonumber \\
-\cos\phi_j Q_{j,k,t} + Q_{j,k,y}-\frac{\sin^2\phi_j}{T\cos\phi_j} Q_{j,k}.\label{edgej}
\end{eqnarray} 

Nulla sed arcu arcu. Duis et ante gravida orci venenatis tincidunt. Fusce vitae lacinia metus. Pellentesque habitant morbi. $\mathbf{A}\underline{\xi}=\underline{\beta}$ Vim $\underline{\xi}$ enum nidi $3(P+2)^{2}$ lacina. Id feugain $\mathbf{A}$ nun quis; magno.

%----------------------------------------------------------------------------------------
%	RESULTS 
%----------------------------------------------------------------------------------------

\section*{Results}

Donec faucibus purus at tortor egestas eu fermentum dolor facilisis. Maecenas tempor dui eu neque fringilla rutrum. Mauris \emph{lobortis} nisl accumsan. Aenean vitae risus ante.
%
\begin{wraptable}{l}{12cm} % Left or right alignment is specified in the first bracket, the width of the table is in the second
\begin{tabular}{l l l}
\toprule
\textbf{Treatments} & \textbf{Response 1} & \textbf{Response 2}\\
\midrule
Treatment 1 & 0.0003262 & 0.562 \\
Treatment 2 & 0.0015681 & 0.910 \\
Treatment 3 & 0.0009271 & 0.296 \\
\bottomrule
\end{tabular}
\captionof{table}{\color{Green} Table caption}
\end{wraptable}
%
Phasellus imperdiet, tortor vitae congue bibendum, felis enim sagittis lorem, et volutpat ante orci sagittis mi. Morbi rutrum laoreet semper. Morbi accumsan enim nec tortor consectetur non commodo nisi sollicitudin. Proin sollicitudin. Pellentesque eget orci eros. Fusce ultricies, tellus et pellentesque fringilla, ante massa luctus libero, quis tristique purus urna nec nibh.

Nulla ut porttitor enim. Suspendisse venenatis dui eget eros gravida tempor. Mauris feugiat elit et augue placerat ultrices. Morbi accumsan enim nec tortor consectetur non commodo. Pellentesque condimentum dui. Etiam sagittis purus non tellus tempor volutpat. Donec et dui non massa tristique adipiscing. Quisque vestibulum eros eu. Phasellus imperdiet, tortor vitae congue bibendum, felis enim sagittis lorem, et volutpat ante orci sagittis mi. Morbi rutrum laoreet semper. Morbi accumsan enim nec tortor consectetur non commodo nisi sollicitudin.

\begin{center}\vspace{1cm}
\includegraphics[width=0.8\linewidth]{placeholder}
\captionof{figure}{\color{Green} Figure caption}
\end{center}\vspace{1cm}

In hac habitasse platea dictumst. Etiam placerat, risus ac.

Adipiscing lectus in magna blandit:

\begin{center}\vspace{1cm}
\begin{tabular}{l l l l}
\toprule
\textbf{Treatments} & \textbf{Response 1} & \textbf{Response 2} \\
\midrule
Treatment 1 & 0.0003262 & 0.562 \\
Treatment 2 & 0.0015681 & 0.910 \\
Treatment 3 & 0.0009271 & 0.296 \\
\bottomrule
\end{tabular}
\captionof{table}{\color{Green} Table caption}
\end{center}\vspace{1cm}

Vivamus sed nibh ac metus tristique tristique a vitae ante. Sed lobortis mi ut arcu fringilla et adipiscing ligula rutrum. Aenean turpis velit, placerat eget tincidunt nec, ornare in nisl. In placerat.

\begin{center}\vspace{1cm}
\includegraphics[width=0.8\linewidth]{placeholder}
\captionof{figure}{\color{Green} Figure caption}
\end{center}\vspace{1cm}

%----------------------------------------------------------------------------------------
%	CONCLUSIONS
%----------------------------------------------------------------------------------------

%% \color{SaddleBrown} % SaddleBrown color for the conclusions to make them stand out

%% \section*{Conclusions}

%% \begin{itemize}
%% \item Pellentesque eget orci eros. Fusce ultricies, tellus et pellentesque fringilla, ante massa luctus libero, quis tristique purus urna nec nibh. Phasellus fermentum rutrum elementum. Nam quis justo lectus.
%% \item Vestibulum sem ante, hendrerit a gravida ac, blandit quis magna.
%% \item Donec sem metus, facilisis at condimentum eget, vehicula ut massa. Morbi consequat, diam sed convallis tincidunt, arcu nunc.
%% \item Nunc at convallis urna. isus ante. Pellentesque condimentum dui. Etiam sagittis purus non tellus tempor volutpat. Donec et dui non massa tristique adipiscing.
%% \end{itemize}

%% \color{DarkSlateGray} % Set the color back to DarkSlateGray for the rest of the content

%----------------------------------------------------------------------------------------
%	FORTHCOMING RESEARCH
%----------------------------------------------------------------------------------------

%% \section*{Forthcoming Research}

%% Vivamus molestie, risus tempor vehicula mattis, libero arcu volutpat purus, sed blandit sem nibh eget turpis. Maecenas rutrum dui blandit lorem vulputate gravida. Praesent venenatis mi vel lorem tempor at varius diam sagittis. Nam eu leo id turpis interdum luctus a sed augue. Nam tellus.

 %----------------------------------------------------------------------------------------
%	REFERENCES
%----------------------------------------------------------------------------------------

\nocite{*} % Print all references regardless of whether they were cited in the poster or not
\bibliographystyle{plain} % Plain referencing style
\bibliography{sample} % Use the example bibliography file sample.bib

%----------------------------------------------------------------------------------------
%	ACKNOWLEDGEMENTS
%----------------------------------------------------------------------------------------

%% \section*{Acknowledgements}

%% Etiam fermentum, arcu ut gravida fringilla, dolor arcu laoreet justo, ut imperdiet urna arcu a arcu. Donec nec ante a dui tempus consectetur. Cras nisi turpis, dapibus sit amet mattis sed, laoreet.

%% %----------------------------------------------------------------------------------------

\end{multicols}
\end{document}
