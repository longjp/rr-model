\documentclass[12pt]{article}
\usepackage{verbatim, amsmath,amssymb,amsthm,graphicx}
\usepackage[margin=1in,nohead]{geometry}
\usepackage{sectsty}
\usepackage{float}
\usepackage{natbib}
\usepackage[usenames]{color}
\usepackage{xspace}
\usepackage{subfig}
\usepackage{bbm}
\usepackage{bm}

\sectionfont{\normalsize}
\subsectionfont{\small}

\title{}
\date{}
\author{}
\newtheorem{theorem}{Theorem}[section]
\newtheorem{definition}{Definition}[section]
\newtheorem{example}{Example}[section]

\newcommand{\Remark}[2]{{\color{red}Remark from #1: #2}\xspace}
\newcommand{\argmin}[1]{\underset{#1}{\operatorname{argmin}}\text{ }}
\newcommand{\argmax}[1]{\underset{#1}{\operatorname{argmax}}\text{ }}
\newcommand{\minimax}[2]{\argmin{#1}\underset{#2}{\operatorname{max}}}
\newcommand{\bb}{\textbf{b}}

\newcommand{\Var}{\text{Var }}
\newcommand{\Cov}{\text{Cov }}
\newcommand{\E}{\mathbb{E}}
\newcommand{\ind}{\mathbbm{1}}

\newcommand{\V}[1]{{\bm{\mathbf{\MakeLowercase{#1}}}}} % vector
\newcommand{\M}[1]{{\bm{\mathbf{\MakeUppercase{#1}}}}} % matrix

\newcommand{\todo}[1]{{\color{red}TODO: #1}}

\newenvironment{my_enumerate}{
  \begin{enumerate}
    \setlength{\itemsep}{1pt}
    \setlength{\parskip}{0pt}
    \setlength{\parsep}{0pt}}{\end{enumerate}
}



% Alter some LaTeX defaults for better treatment of figures:
% See p.105 of [yas] elisp error!TeX Unbound'' for suggested values.
% See pp. 199-200 of Lamport's [yas] elisp error!LaTeX'' book for details.
%   General parameters, for ALL pages:
\renewcommand{\topfraction}{0.9}% max fraction of floats at top
\renewcommand{\bottomfraction}{0.8}% max fraction of floats at bottom
%   Parameters for TEXT pages (not float pages):
\setcounter{topnumber}{2}
\setcounter{bottomnumber}{2}
\setcounter{totalnumber}{4}     % 2 may work better
\setcounter{dbltopnumber}{2}    % for 2-column pages
\renewcommand{\dbltopfraction}{0.9}% fit big float above 2-col. text
\renewcommand{\textfraction}{0.07}% allow minimal text w. figs
%   Parameters for FLOAT pages (not text pages):
\renewcommand{\floatpagefraction}{0.7}% require fuller float pages
% N.B.: floatpagefraction MUST be less than topfraction !!
\renewcommand{\dblfloatpagefraction}{0.7}% require fuller float pages

% remember to use [htp] or [htpb] for placement





\begin{document}
\noindent
\textbf{James Long}\\
\textbf{\today}\\
\textbf{Uncertainty Quantification on Period Estimates}

Figure \ref{fig:data} shows an example of a star which varies in brightness periodically with time. The period of variation is not clear from this data.

\begin{figure}[H]
  \begin{center}
    \begin{includegraphics}[scale=0.4]{data.png}
      \caption{\label{fig:data}}
    \end{includegraphics}
  \end{center}
\end{figure}

We compute the residual sum of squares at a set of frequences $\omega$.
\begin{equation*}
  RSS(\omega) = \min_{a,\phi,\beta_0} \sum_{i=1}^n (y_i - a\sin(\omega t_i + \phi) - \beta_0)^2
\end{equation*}
The frequency estimate is
\begin{equation*}
  \widehat{\omega} = \argmin{\omega} RSS(\omega)
\end{equation*}

The RSS as a function of $\omega$ is plotted in Figure \ref{fig:rss}. The red vertical line is $\widehat{\omega}$. This frequency appears to fit much better than any others.

  \begin{figure}[H]
    \begin{center}
      \begin{includegraphics}[scale=0.4]{rss.png}
        \caption{\label{fig:rss}}
      \end{includegraphics}
    \end{center}
  \end{figure}
  

  The period estimate is $p = 2\pi/\widehat{\omega} = 0.5134$. We can ``fold'' the original data in Figure \ref{fig:data} on the period estimate, i.e. plot phase instead of time. The resulting plot is Figure \ref{fig:data_folded}. Now the pattern of brightness variation is clear.

  \begin{figure}[H]
  \begin{center}
    \begin{includegraphics}[scale=0.4]{data_folded.png}
      \caption{\label{fig:data_folded}}
    \end{includegraphics}
  \end{center}
\end{figure}


  The example above is very easy because the data is so high quality. More realistically we might observe the function in Figure \ref{fig:data} 15 times. We sample 15 observations from the function and compute $RSS(\omega)$. We repeat this process 20 times. In Figure \ref{fig:rss_down} we plot these 20 RSS functions all on top of each other. The max RSS for each function has been normalized to $0$. The 20 vertical yellow lines are the 20 $\widehat{\omega}$ values. 8 in 20 times the $\widehat{\omega}$ is about correct and 12 in 20 times it is way off. There is no clear pattern to the errors. The red line is the median (across the 20 runs) of the RSS at each frequency. There do not appear to be many preferred frequencies other than the truth, e.g. I don't see clear aliasing.
  

  \begin{figure}[H]
    \begin{center}
      \begin{includegraphics}[scale=0.4]{rss_down.png}
        \caption{\label{fig:rss_down}}
      \end{includegraphics}
    \end{center}
  \end{figure}
  

  Specializing to one run we plot the RSS function for the frequency estimate near $7$. We can see that the true frequency (vertical red line) also fits the data quite well (perhaps the third best fitting frequency). I would like some form of uncertainty quantification which captures this fact.

  \begin{figure}[H]
    \begin{center}
      \begin{includegraphics}[scale=0.4]{rss_one_down.png}
        \caption{\label{fig:rss_one_down}}
      \end{includegraphics}
    \end{center}
  \end{figure}
  
  

  


%\bibliographystyle{plainnat}
%\bibliography{refs}


\end{document}

